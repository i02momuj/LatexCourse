%Document class
\documentclass{article}

%Document content
\begin{document}
El \'area se calcula como $2 \pi r$.

$2 + 3 = 5$

$2 \times 3 = 6$

$(2+3) \cdot 2 = 10$

$6 \div 2 = 3$

Las fracciones se representan como $\frac{2x}{3}$

\begin{equation}
\frac{2 \cdot x}{5}
\end{equation}

$x_{a,b}$

$x^2$

$(x_{1}+x_{2})^2=x_{1}^{2}+2x_{1}x_{2}+x_{2}^{2}$

-

$\left(\frac{\frac{2}{3}+\frac{1}{2}}{\frac{1}{2}}\right)=\frac{7}{6}$

$f(x,y)=\sqrt{\frac{x+y}{x-y}}$

$\sum_{i=1}^{n}{2^i}$

$\frac{\partial }{\partial x}$

$\int_{a}^{b} \! f(x)  \,dx$

\end{document}
