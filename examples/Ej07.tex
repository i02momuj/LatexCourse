%Document class
\documentclass{article}

%Document content
\begin{document}

Lista simple
\begin{itemize}
  \item Elemento 1
  \item Elemento 2
  \item[+] Elemento 3
  \item[=] Elemento 4
  \item[-] Elemento 5
\end{itemize}

\hfill \break

Lista numerada
\begin{enumerate}
  \item Elemento 1
  \item Elemento 2
  \item Elemento 3
  \begin{enumerate}
    \item Elemento 3a
    \item Elemento 3b
    \item Elemento 3c
  \end{enumerate}
\end{enumerate}

\hfill \break

Lista descriptiva
\begin{description}
  \item [Elemento 1] A continuacion se incluye la descripci\'on completa del elemento nº 1. Se encuentra dentro del entorno description. La descripcion de los elementos se ve de manera diference a una lista normal, pero es muy intuitiva.
  \item [Elemento 2] A continuacion se incluye la descripci\'on completa del elemento numero 2. Se encuentra dentro del entorno description.
\end{description}

\end{document}
